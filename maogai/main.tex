\documentclass[UTF8]{ctexart}
\usepackage{graphicx}
\usepackage{fancyhdr}
\usepackage{titlesec}
\usepackage{titletoc}
\usepackage{appendix}
\usepackage{bm, amsmath,amsfonts}
\usepackage{multirow}
\usepackage{enumerate}
\usepackage[a4paper,left=3.4cm,right=3cm,top=1.65cm,bottom=2.54cm]{geometry}
\renewcommand{\contentsname}{\zihao{3} 目\quad 录}

%页眉页脚设置
\pagestyle{fancy}
\fancyhf{}
\cfoot{\thepage}
\lhead{\kaishu~毛概复习~}
\chead{\kaishu~电气153~}
\rhead{\kaishu~刘玉林~}
%目录页设置
\titlecontents{section}[0em]{\zihao{4}\bf }{\thecontentslabel\ }{}
{\hspace{.5em}\titlerule*[4pt]{$\cdot$}\contentspage}
\titlecontents{subsection}[2em]{\vspace{0.1\baselineskip}\zihao{-4}}{\thecontentslabel\ }{}
{\hspace{.5em}\titlerule*[4pt]{$\cdot$}\contentspage}
\titlecontents{subsubsection}[4em]{\vspace{0.1\baselineskip}\zihao{-4}}{\thecontentslabel\ }{}
{\hspace{.5em}\titlerule*[4pt]{$\cdot$}\contentspage}

%------------------------------------------------------------------------
%正文部分
\begin{document}
    \begin{titlepage}
    \centering
    \centerline{\includegraphics[width=14cm,height=2.54cm]{1.jpg}}
    \vspace*{\stretch{1}}
    {\sffamily\fontsize{20}{29} 毛泽东思想和中国特色社会主义理论体系概论重点}\\
    \vspace{\stretch{1}}
    {\Large {\fangsong 电气153刘玉林}\\[5pt]\texttt{1809040656@qq.com}}\\
    \vspace{\stretch{1}}
    {\today}
    \end{titlepage}
\tableofcontents\thispagestyle{empty}
\newpage 
\setcounter{page}{1}
\section{第一章}
\subsection{毛泽东思想成熟的标志。}
新民主主义革命理论的系统阐述,实现了马克思主义与中国革命实践相结合的历史性的飞跃,标志着毛泽东思想得到多方面展开而趋于成熟。
1945年党的七大将毛泽东思想写入党章,确立为党必须长期坚持的指导思想。
\subsection{毛泽东思想活的灵魂。}
\begin{enumerate}[(1)]
\item 实事求是
\par 就是一切从实际出发,理论联系实际,坚持在实践中检验真理和发展真理。
\item 群众路线 
\par 群众路线,就是一切为了群众,一切依靠群众,从群众中来,到群众中去,把党的正确主张变为群众的自主行动。
\item 独立自主
\par 独立自主,就是坚持独立思考,走自己的路,就是坚定不移地维护民族独立、捍卫国家主权,把立足点放在依靠自己力量的基础上,同时积极争取外援,开展国际经济文化交流,学习外国一切对我们有益的先进事物。
\end{enumerate}
\subsection{毛泽东是马克思主义中国化的伟大开拓者,是毛泽东思想的主要创立者。在中国共产党历史上,毛泽东第一个明确提出了“马克思主义中国化”的科学命题和重大任务。p14(4-1)}
\par 深刻论证了马克思主义中国化的必要性和极端重要性,系统阐述了马克思主义中国化的科学内涵和实现马克思主义中国化的正确途径,开辟了马克思主义在中国发展的宽广道路,为党领导的革命和建设事业的发展奠定了坚实的思想理论基础。
\subsection{毛泽东思想是马克思主义中国化第一次历史性飞跃的理论成果。p15(1-5)}
毛泽东思想在新民主主义革命、社会主义革命和建设,革命军队建设、革命战略和国防建设,政策和策略,思想政治工作和文化工作,外交工作和党的建设等方面,以独创性的理论丰富和发展了马克思列宁主义。
\subsection{在马克思主义中国化的历史进程中,毛泽东思想为中国特色社会主义理论体系的形成奠定了理论基础。p15(2-1)}
\par 尤其是毛泽东思想关于社会主义建设的理论,为开设和发展中国特色社会主义作了重要的理论准备。
\subsection{毛泽东一生为党和人民的事业作出了杰出贡献,《关于建国以来党的若干历史问题的决议》对毛泽东和毛泽东思想的历史地位作出了科学的、实事求是的评价,对于统一全党的认识起到了重要作用,得到了全党的拥护。p15(2-1)}
\section{第二章}
\subsection{近代中国国情。}
\par 鸦片战争后,由于西方列强的入侵,由于封建统治的腐败,中国逐渐成为半殖民地半封建社会。
\subsection{近代中国革命的时代特征。p21(2-1)}
\par 近代中国的社会性质和主要矛盾,决定了中国革命是资产阶级民主革命。
\par 经历了从旧民主主义向新民主主义革命阶段的转变。
\subsection{以五四运动的爆发为标志,中国资产阶级民主革命进入新民主主义革命的崭新阶段。p21(4-1)}
\subsection{新民主主义革命的总路线。p24(3-4)}
\par 即无产阶级领导的,人民大众的,反对帝国主义、封建主义和官僚资本主义的革命。
\subsection{新民主主义革命的对象、动力、领导力量、性质和前途。p25-30}
\begin{enumerate}[(1)]
\item 革命的对象
\par 从总体上说,中国革命的对象是帝国主义、封建主义和官僚资本主义。
\item 革命的动力
\par 无产阶级、农民阶级、城市小资产阶级、民族资产积极。
\item 革命的领导力量
\par 中国无产阶级及其政党
\item 革命的性质和前途
\par 新民主主义革命仍然属于资产阶级民主主义革命的范畴。它推翻帝国主义、封建主义和官僚资本主义的反动统治,在政治上争取和联合民族资产阶级去反对共同的敌人,在经济上保护民族工商业,容许有利于国计民生的私人资本主义发展。它要建立的是无产阶级领导的各革命阶级的联合专政,而不是无产阶级专政。        
\end{enumerate}
\subsection{新民主主义的政治纲领、经济纲领、文化纲领。p31-33}
\begin{enumerate}[(1)]
\item 政治纲领
\par 推翻帝国主义和封建主义的统治,建立一个无产阶级领导的、以工农联盟为基础的、各革命阶级联合专政的新民主主义的共和国。
\item 经济纲领
\par 没收封建地主阶级的土地归农民所有,没收官僚资产阶级的垄断资本归新民主主义的国家所有,保护民族工商业。
\item 文化纲领
\par 无产阶级领导的人民大众的反帝反封建的文化,即民族的科学的大众的文化。
\end{enumerate}
\subsection{新民主主义革命道路提出的必然性。p34}
\par 国情决定。近代中国是一个半殖民地半封建社会,内无民主制度而受封建制度的压迫;外无民族独立而受帝国主义的压迫。中国革命的主要斗争形势是武装斗争。
\par 近代中国农民占全国人口的绝大多数,是无产阶级可靠的同盟军和革命的主力军。
\par 敌人建立了庞大的反革命军队,并长期占据着中心城市,而农村则是其统治的薄弱环节。
\subsection{新民主主义革命三大法宝及其相互关系。p36-41}
\par 统一战线、武装斗争、党的建设是三大法宝。
\par 关系:  统一战线和武装斗争层中国革命的两个基本特点、是战胜敌人的两个基本武器。统一战线是实行武装斗争的统一战线,武装斗争是统一一战线的中心支柱,党的组织则是掌握统一战线和武装斗争这两个武器以实行对敌冲锋陷阵的英勇战士。
\subsection{中国革命分几步走。p22}
\par 两步,第一步是完成反帝反封建的新民主主义革命任务,第二步是完成社会主义革命任务,这是性质不同但又相互联系的两个革命过程。
\section{第三章}
\subsection{1956年三大改造的基本完成标志着中国历史上长达数千年的阶级剥削制度的结束和社会主义基本制度的确立。p56}
\subsection{新民主主义社会的性质p44}
\subsection{在新民主主义社会中,存在着五种经济成分,即社会主义性质的国营经济、半社会主义性质的合作社经济、农民和手工业者的个体经济、私人资本主义经济和国家资本主义经济。其中半社会主义性质的合作社经济是个体经济向社会主义集体经济过渡的形式,国家资本主义经济是私人资本主义经济向社会主义国营经济过渡的形式。所以,主要的经济成分是三种:社会主义经济、个体经济和资本主义经济。p44}
\subsection{中国共产党确立了我国由新民主主义社会向社会主义过渡的总路线.p48}
\subsection{农业社会主义改造步骤。p51}
\begin{enumerate}[(1)]
\item  积极引导农民组织起来,走互助合作道路。
\item  遵循自愿互利、典型示范和国家帮助的原则,以互助合作的优越性吸引农民走互助合作道路。
\item  正确分析农民的阶级状况,制定正确的阶级政策。
\item  坚持积极领导、稳步前进的方针,采取循序渐进的步骤。
\end{enumerate}
\subsection{手工业社会主义改造的步骤。p52}
\par 党和政府采取了积极引导、稳步前进的方针。
\begin{enumerate}[(1)]
    \item 办手工业供销小组
    \item 办手工业供销合作社
    \item 建立手工业生产合作社
\end{enumerate}
\subsection{资本主义工商业社会主义改造。p53-55}
\begin{enumerate}[(1)]
    \item 用和平赎买的方法改造资本主义工商业
    \item 采取从低级到高级的的国家资本主义的过渡形式
    \item 把资本主义工商业者改造成为自食其力的社会主义劳动者
\end{enumerate}
\subsection{我国进行社会主义改造的历史经验。p56}
\begin{enumerate}[(1)]
    \item 坚持社会主义工业化建设与社会主义改造同时并举
    \item 采取积极引导、逐步过渡的方式
    \item 用和平的方法进行改造
\end{enumerate}
\section{第四章}
\subsection{1956年4月和5月,毛泽东先后在中央政治局扩大会议和最高国务会议上,作了《论十大关系》的报告,初步总结了我国社会主义建设的经验,明确提出要以苏为鉴,独立自主地探索适合中国情况的社会主义建设道路。p65}
\subsection{论十大关系的基本方针。p66}
\par "努力把党内党外,国内国外的一切积极的因素,直接的、间接的积极因素全部调动起来"
\subsection{社会主义社会的基本矛盾是在生产关系和生产力基本适应、上层建筑和经济基础基本适应条件下的矛盾,是在人民根本利益一致基础上的矛盾。因此,它不是对抗性的矛盾,而是非对抗性的矛盾。p70}
\subsection{“我们历来就主张,在人民民主专政下面,解决敌我之间的和人民内部的这两类不同性质的矛盾,采用专政和民主这样两种不同的方法。” 所谓专政方法,就是运用人民民主专政的国家机器,对于国家内部那些反抗社会主义改造、破坏社会主义建设的敌对分子和严重犯罪分子依法治罪,剥夺他们的政治权利,强迫他们从事劳动,并在劳动中尽量使他们改造成为新人。所谓民主方法,就是讨论的方法、批评的方法、说服教育的方法。p71-72}
\subsection{新中国刚刚建立的时候,我国的工业基础非常薄弱,在很多工业领域甚至还是空白。对此,党确定了把实现国家工业化作为新中国整个经济建设的主要任务。p73}
\subsection{社会主义建设初步探索的意义和经验教训。p76-79}
\par 经验:
\begin{enumerate}[(1)]
    \item 巩固和发展了我国的社会主义制度
    \item 为开创中国特色社会主义提供了宝贵的经验、理论准备、物质基础
    \item 丰富了科学社会主义的理论和实践
\end{enumerate}
\par 教训:
\begin{enumerate}[(1)]
    \item 必须把马克思主义与中国实际相结合,探索符合中国特点都社会主义建设道路
    \item 必须正确认识社会主义社会的主要矛盾和根本任务,集中力量发展生产力
    \item 必须从实际出发进行社会主义建设,建设规模和速度要和国力相适应,不能急于求成
\end{enumerate}
\section{第五章}
\subsection{如何认识邓小平理论形成的社会历史条件?p87}
\begin{enumerate}[(1)]
    \item 和平与发展成为时代主题是邓小平理论形成的时代背景
    \item 社会主义建设的经验教训是邓小平理论形成的历史根据
    \item 改革开放和现代化建设的实践是邓小平理论形成的现实依据
\end{enumerate}
\subsection{什么是社会主义,怎样建设社会主义是邓小平思考的首要的基本的理论问题。p91}
\subsection{1992年初,邓小平在南方谈话中对社会主义本质作了理论概括:“社会主义的本质,是解放生产力,发展生产力,消灭剥削,消除两极分化,最终达到共同富裕。”p92 }
\subsection{十一届届三中全会, 重新确立了解放思想、实事求是的思想路线,停止使用“以阶级斗争为纲”的错误提法,确定把全党工作的重点转移到现代化建设上来,作出实行改革开放的重大决策。p90}
\subsection{十三大报告第一次系统论述了社会主义初级阶段理论。报告指出,社会主义初级阶段,是从我国进入社会主义到基本实现社会主义现代化的整个历史阶段。 至少一百年时间。p91(后面那句书上没)}
\subsection{社会主义初级阶段的两层含义p99}
 \par 第一,我国已经进入社会主义社会,必须坚持而不能离开社会主义。
 \par 第二,我国的社会主义社会还处在不发达的阶段,必须正视而不能超越初级阶段。
\subsection{党在社会主义初级阶段总路线:领导和团结全国各族人民,以经济建设为中心,坚持四项基本原则,坚持改革开放,自力更生,艰苦创业,为把我国建设成为富强、民主、文明、和谐、美丽的社会主义现代化强国而奋斗。p100}
\subsection{社会主义的根本任务是发展生产力。p102}
\subsection{党的十三大把邓小平“三步走”的发展战略构想确定下来,明确提出:第一步,从1981年到1990年实现国民生产总值比1980年翻一番,解决人民的温饱问题;第二步,从1991年到20世纪末,使国民生产总值再翻一番,达到小康水平;第三步,到21世纪中叶,国民生产总值再翻两番,达到中等发达国家水平,基本实现现代化。然后在这个基础上继续前进。p103-104}
\subsection{改革的性质:是社会主义制度的自我完善和发展。p105}
\subsection{改革是社会主义社会发展的直接动力。p105}
\subsection{判断改革和各方面工作的是非得失,归根到底,要以是否有利于发展社会主义社会的生产力,是否有利于增强社会主义国家的综合国力,是否有利于提高人民的生活水平为标准。p106}
\subsection{改革开放开始后很长时期内,经济体制改革的核心问题是如何正确认识和处理计划与市场的关系。p107}
\subsection{党的十四大确立了建立社会主义市场经济体制的改革目标。p108}
\newpage
\subsection{社会主义市场经济理论的要点有:p108}
\begin{enumerate}[(1)]
    \item 计划经济和市场经济不是划分社会制度的标志,计划经济不等于社会主义,市场经济也不等于资本主义
    \item 计划和市场都是经济手段,对经济活动的调节各有优势和长处,社会主义实行市场经济要把两者结合起来
    \item 市场经济作为资源配置的一种方式本身不具有制度属性,可以和不同的社会制度结合,从而表现出不同的性质。坚持社会主义制度与市场经济的结合,是社会主义市场经济的特色所在。
\end{enumerate}
\subsection{“一国两制”的伟大构想的提出是从台湾问题开始的,在实践中首先被运用、解决了香港问题、澳门问题。}
\subsection{“和平统一、一国两制”构想的基本内容:p111}
\begin{enumerate}[(1)]
    \item 坚持一个中国,这是“和平统一、一国两制”的核心,是发展两岸关系和实现和平统一的基础
    \item 两制并存: 在祖国统一的前提下,国家的主体部分实行社会主义制度,同时在台湾、香港、澳门保持原有的社会制度和生活方式长期不变
    \item 高度自治: 祖国完全统一后,台湾、香港、澳门作为特别行政区,享有不同于中国其他省、市、自治区的高度自治权,台湾、香港、澳门同胞各种合法权益将得到切实尊重和维护; 
    \item 尽最大努力争取和平统一,但不承诺放弃使用武力
    \item 解决台湾问题,实现祖国完全统一,寄希望于台湾人民
\end{enumerate}
\subsection{邓小平理论的历史地位p114}
\begin{enumerate}[(1)]
    \item 马克思列宁主义、毛泽东思想的继承和发展。
    \item 中国特色社会主义理论体系的开篇之作。
    \item 改革开放和社会主义现代化建设的科学指南。
\end{enumerate}
\section{第六章}
\subsection{三个代表重要思想回答和解决的是“建设什么样的党、怎样建设党”的重大问题。p123}
\subsection{进一步提高党的领导水平和执政水平、提高拒腐防变和抵御风险的能力,是我们党必须解决好的两大历史性课题。p121}
\subsection{2002年5月31日,江泽民在中共中央党校省部级干部进修班毕业典礼上深刻阐述了“三个代表”重要思想的内在联系,提出“贯彻‘三个代表’要求,关键在坚持与时俱进,核心在保持党的先进性,本质在坚持执政为民”。p125}
\subsection{十六大将“三个代表”重要思想与马克思列宁主义、毛泽东思想和邓小平理论一道确立为党必须长期坚持的指导思想,并写入党章。p126}
\subsection{“三个代表”重要思想的核心观点“中国共产党必须始终代表中国先进生产力的发展要求,代表中国先进文化的前进方向,代表中国最广大人民的根本利益。”这是我们党的立党之本、执政之基、力量之源。p144}
\subsection{党的十四大正式把建立社会主义市场经济体制确立为我国经济体制改革的目标。p132}
\subsection{社会主义市场经济是同社会主义基本制度结合在一起的,既可以发挥市场经济的长处,又可以发挥社会主义制度的优越性。江泽民强调,要坚持社会主义市场经济的根本方向,使市场在国家宏观调控下对资源配置起基础性作用。p133}
\subsection{建立社会主义市场经济体制,必须坚持和完善公有制为主体、多种所有制经济共同发展的社会主义基本经济制度。必须毫不动摇地巩固和发展公有制经济。必须毫不动摇地鼓励、支持和引导非公有制经济发展。个体、私营等各种形式的非公有制经济是社会主义市场经济的重要组成部分。发挥市场机制的作用和国家宏观调控,都是社会主义市场经济体制的内在要求。p133}
\subsection{江泽民在党的十五大报告中初步勾画了实现第三步战略目标的蓝图:21世纪第一个十年实现国民生产总值比2000年翻一番,使人民的小康生活更加宽裕,形成比较完善的社会主义市场经济体制;再经过十年的努力,到建党一百年时,使国民经济更加发展,各项制度更加完善;到21世纪中叶新中国成立一百年时,基本实现现代化,建成富强民主文明的社会主义国家。p135}
\newpage
\subsection{建设社会主义政治文明,最根本的就是要坚持党的领导、人民当家作主和依法治国的有机统一。这是我们推进政治文明建设必须遵循的基本方针,也是我国社会主义政治文明区别于资本主义政治文明的本质特征。党的领导是人民当家作主和依法治国的根本保证,人民当家作主是社会主义民主政治的本质要求,依法治国是党领导人民治理国家的基本方略。p136}
\subsection{依法治国,就是广大人民群众在党的领导下,依照宪法和法律规定,通过各种途径和形式管理国家事务,管理经济文化事业,管理社会事务,保证国家各项工作都依法进行,逐步实现社会主义民主的制度化、法律化,使这种制度和法律不因领导人的改变而改变,不因领导人看法和注意力的改变而改变。p136}
\subsection{实行依法治国,必须坚持有法可依、有法必依、执法必严、违法必究。p137}
\subsection{坚持党的领导,核心是坚持党的先进性。p138}
\subsection{加强和改进党的作风建设,核心问题是保持党同人民群众的血肉联系。我们党的最大政治优势是密切联系群众,党执政后的最大危险是脱离群众。p139}
\subsection{“三个代表”重要思想的历史地位:p140}
\begin{enumerate}[(1)]
    \item 中国特色社会主义理论体系的接续发展。
    \item 加强和改进党的建设,推进中国特色社会主义事业的强大理论武器。
\end{enumerate}
\section{第七章}
\subsection{2003年10月,党的十六届三中全会通过的《中共中央关于完善社会主义市场经济体制若干问题的决定》,使我们党的文件中第一次提出科学发展观。p149}
\subsection{社会主义核心价值体系:马克思主义指导思想,中国特色社会主义共同理想,以爱国主义为核心的民族精神和以改革创新为核心的时代精神,社会主义荣辱观。p163}
\subsection{社会主义核心价值观:富强、民主、文明、和谐,自由、平等、公正、法治,爱国、敬业、诚信、友善。p163}
\subsection{民主法治、公平正义、诚信友爱、充满活力、安定有序、人与自然和谐相处,是构建社会主义和谐社会的总要求。p163}
\subsection{科学发展观的内涵:}
\begin{enumerate}[(1)]
    \item 推动经济社会发展是科学发展观的第一要义
    \item 以人为本是科学发展观的核心立场
    \item 全面协调可持续是科学发展观的基本要求
    \item 统筹兼顾是科学发展观的根本方法
\end{enumerate}
\section{第八章}
\subsection{社会主要矛盾的变化和依据。p178-179}
\par 变化
\par \qquad 从人民日益增长的物质文化需要同落后的社会生产之间的矛盾变化为人民日益增长的美好生活需要和不平衡不充分的发展之间的矛盾.
\begin{enumerate}[(1)]
    \item 经过改革开放40年的发展,我国社会生产力水平总体上显著提高,很多方面进入世界前列
    \item 人民生活水平显著提高,对美好生活的向往更加强烈,不仅对物质文化生活提出了更高要求,而且在民主、法治、公平、安全、环境等方面的要求日益增长。
    \item 影响满足人民美好生活需求的因素很多,但主要是发展的不平衡,制约了全面发展水平提升。
\end{enumerate}
\subsection{新时代的内涵和意义p181-182}
\par 内涵:
\begin{enumerate}[(1)]
    \item 这个新时代尼承前启后、一排往开来在新的历史条件下继续夺取中国特色社会主义伟大胜利的时代。
    \item 这个新时代是决胜会面建成小康社会、进而全面建设社会主义现代化强国的时代。
    \item 这个新时代是全国各族人民团结奋斗、不断创造美好生活、逐步实现全体人民共同富裕的时代。
    \item 这个新时代是全体中华儿女勤力同心、奋力实现中华民族伟大复兴中国梦的时代。
    \item 这个新时代是我国日益走近世界舞台中央、不断为人类作出更大贡献的时代。
\end{enumerate}
\par 意义:
\begin{enumerate}[(1)]
    \item 从中华民族复兴的历史进程看,中国特色社会主义进人时代,意味着近代以来久经磨难的中华民族迎来了从站起来、富起来到强起来的伟大飞跃
    \item 从科学社会主义发展进程看,中国特色社会主义进入新时代,意味着科学社会主义在21世纪的中国焕发出强大生机活力,在世界上高高举起了中国特色社会主义伟大旗帜。
    \item 从人类文明进程看,中国特色社会主义进入新时代,意味着中国特色社会主义道路、理论、制度、文化不断发展,拓展了发展中国家走向现代化的途径
\end{enumerate}
\subsection{习近平新时代中国特色社会主义思想的核心要义和丰富内涵P184}
\par 核心要义:
\par \quad 坚持和发挥中国特色社会主义
\par 丰富内涵:
\begin{enumerate}[(1)]
    \item 明确坚持和发展中国特色社会主义,总任务是实现社会主义现代化和中华民族伟大复兴、在全面建成小康社会的基础上,分两步走在本世纪中叶建成富强民主文明和谐美丽的社会主义现代化强国。
    \item 明确新时代我国社会主要矛盾是人民日益增长的美好生活需要和不平衡不充分的发展之间的矛盾,必须坚持以人民为中心的发展思想,不断促进人的全面发展、全体人民共同高裕。
    \item 明确中国特色社会主义事业总体布局是“五位一体”、战略布局是“四个全面”,强调坚定道路自信、理论自信、制度自信、文化自信。
    \item 明确全面深化改革总目标是完善和发展中国特色社会主义制度、推进国家治理体系和治理能力现代化。
    \item 明确全面推进依法治国总目标是建设中国特色社会主义法治体系、建设社会主义法治国家。
    \item 明确党在新时代的强车目标是建设一支听党指挥、能打胜仗、作风优良的人民军队,把人民军队建设成为世界流军队。
    \item 明确中国特色大国外交要推动构建新型国际关系,推动构建人类命运共同体。
    \item 明确中国特色社会主义最本质的特征是中国共产党领导,中国特色社会主义制度的最大优势是中国共产党领导,党是最高政治领导力量,提出新时代党的建设总要求,突出政治建设在党的建设中的重要地位。

\end{enumerate}
\subsection{坚持和发展中国特色社会主义的基本方略P185-188}
\begin{enumerate}[(1)]
    \item 坚持党对一切工作的领导
    \item 坚持以人民为中心
    \item 坚持全面深化改革
    \item 坚持新发展理念
    \item 坚持人民当家作主
    \item 坚持全面依法治国
    \item 坚持社会主义核心价值体系
    \item 坚持在发展中保障和改善民生
    \item 坚持人与自然的和谐共生
    \item 坚持总体国家安全观
    \item 坚持党对人民军队的绝对领导
    \item 坚持一国两制和推进祖国统一
    \item 坚持推动构建人类民运共同体
    \item 坚持全面从严治党

\end{enumerate}
\subsection{习近平新时代中国特色社会主义思想的历史地位P189-191}
是马克思主义中国化最新成果,是中国特色社会主义理论体系的重要组成部分,是当代中国马克思主义,21世纪马克思主义,是党和国家必须长期坚持并不断发展的指导思想,是全党全国人民为实现中华民族伟大复兴而奋斗的行动指南
\section{第九章}
\subsection{坚持和发展中国特色社会主义的总任务,是实现社会主义现代化和中华民族伟大复兴,在全面建成小康社会的基础上,分两步走在本世纪中叶建成富强民主文明和谐美丽的社会主义现代化强国。p195}
\subsection{中国梦的本质内容是:国家富强、民族振兴、人民幸福p197}
\subsection{习近平指出:“实现中国梦必须走中国道路、弘扬中国精神、凝聚中国力量。”p199}
\subsection{“中国梦”归根到底是人民的梦,是中华民族伟大复兴的形象表达。 p198后半句没}
\subsection{习近平在党的十九大报告中提出,全面建设社会主义现代化国家的进程分两个阶段来安排。第一个阶段,从2020年到2035年,在全面建成小康社会的基础上,再奋斗15年,基本实现社会主义现代化。第二个阶段,从2035年到本世纪中叶,在基本实现现代化的基础上,再奋斗15年,把我国建成富强民主文明和谐美丽的社会主义现代化强国。p203}
\subsection{中国梦的科学内涵、相互关系和特点。实现中华民族伟大复兴的中国梦的途径。}
\par 国家富强、民族振兴是人民幸福的基础和保障。
\par 人民幸福是国家富强,民族振兴的题中之义和必然要求
\par 人民幸福是国家富强,民族振兴的根本出发点和落脚点
\section{第十章}
\subsection{新发展理念的提出与内容。p207}
党的十八届五中全会坚持以人民为中心的发展思想,鲜明提出了创新、协调、绿色、开放、共享的新发展理念。创新是引领发展的第一动力。协调是持续健康发展的内在要求。绿色是永续发展的必要条件。开放是国家繁荣发展的必由之路。共享是中国特色社会主义的本质要求。创新、协调、绿色、开放、共享的发展理念,相互贯通、相互促进,是具有内在联系的集合体。
\subsection{建设现代化经济体系,当前要突出抓好哪几个方面工作p212-214}
现代化经济体系,是由社会经济活动各个环节、各个层面、各个领域的相互关系和内在联系构成的一个有机整体。建设现代化经济体系,当前要突出抓好:第一,大力发展实体经济。第二,加快实施创新驱动发展战略。第三,激发各类市场主体活力。第四,积极推动城乡区域协调发展。第五,着力发展开放型经济。第六,深化经济体制改革。
\subsection{中国特色社会主义政治发展道路的内容。p215}
走中国特色社会主义政治发展道路,必须坚持党的领导、人民当家作主、依法治国有机统一。必须坚持正确政治方向。必须深化机构和行政体制改革。当前我国正在根据《中共中央关于党和国家机构改革的方案》进行全方位的改革。

\subsection{人民当家作主制度体系包括哪些内涵。p216}
人民代表大会制度是我国根本政治制度,是符合中国国情、体现中国社会主义国家性质、能够保证人民当家作主的根本政治制度和最高实现形式;发挥社会主义协商民主重要作用,协商民主是中国社会主义民主政治的特有形式和独特优势,是实现党的领导的重要方式;中国共产党领导的多党合作和政治协商制度是我国的一项基本政治制度,人民政协是具有中国特色的制度安排,是社会主义协商民主的重要渠道和专门协商机构;民族区域自治制度是我国的一项基本政治制度,是中国特色解决民族问题正确道路的重要内容和制度保障;基层群众自治制度是我国的一项基本政治制度。完善基层群众自治制度,发展基层民主,是社会主义民主政治建设的基础。
\subsection{如何准确把握“一国”和“两制”的关系。p220}
“一国两制”是一个完整的概念。“一国”是实行“两制”的基础,“两制”从属和派生于“一国”,并统一于“一国”之中。“一国”是根,根深才能叶茂;“一国”是本,本固才能枝荣。必须牢固树立“一国”意识,坚守“一国”原则,正确处理特别行政区和中央的关系。与此同时,在“一国”的基础之上,“两制”的关系应该也完全可以做到和谐相处、相互促进。
\subsection{如何把握文化自信与实现中华民族伟大复兴的关系?p230}
没有高度的文化自信,没有文化的繁荣兴盛,就没有中华民族伟大复兴。坚定文化自信,推动社会主义文化繁荣兴盛,实现中华民族伟大复兴,要以马克思主义为指导,坚守中华文化立场,立足当代中国现实,结合当今时代条件,发展面向现代化、面向世界、面向未来的,民族的科学的大众的社会主义文化,走中国特色社会主义文化发展道路,建设中国特色社会主义文化。
\subsection{如何牢牢掌握意识形态工作领导权?p223-224}
意识形态关乎旗帜、关乎道路、关乎国家政治安全,决定文化前进方向和道路。建设中国特色社会主义文化,必须建设具有强大凝聚力、引领力的社会主义意识形态,使全体人民在理想信念、价值理念、道德观念上紧紧团结在一起,巩固马克思主义在意识形态领域的指导地位,牢牢掌握意识形态工作领导权。
掌握意识形态工作领导权,要旗帜鲜明坚持马克思主义指导地位。要加快构建中国特色哲学社会科学。要坚持正确的舆论导向。要建设好网络空间。要落实好意识形态工作责任制。

\subsection{社会主义核心价值体系与社会主义核心价值观的内容。p226从下向上}
社会主义核心价值体系由马克思主义指导思想、中国特色社会主义共同理想、以爱国主义为核心的民族精神和以改革创新为核心的时代精神、社会主义荣辱观四个方面内容构成。社会主义核心价值观是社会主义核心价值体系的内核凝练和集中表达,体现着社会主义核心价值体系的根本性质和基本特征,反映着社会主义核心价值体系的丰富内涵和实践要求。包括、民主、文明、和谐,自由、民主、公正、法治,爱国、敬业、诚信、友善的24字表达,把涉及国家、公民、社会三个层面的价值要求融为一体。
\subsection{如何培育和践行社会主义核心价值观。p227}
\begin{enumerate}[(1)]
    \item 把社会主义核心价值观融入到社会生活各个方面
    \item 坚持全民行动,干部带头,从家庭做起,从娃娃抓起
    \item 立足中华优秀传统文化和革命文化
    \item 发扬中国人民在长期奋斗中培育,继承,发展起来的伟大民族精神。
\end{enumerate}
\subsection{当前如何保障和改善民生水平。p231-233}
\begin{enumerate}[(1)]
    \item 优先发展教育事业
    \item 提供就业质量和人民收入水平
    \item 加强社会保障体系建设
    \item 坚决打赢拖贫攻坚战
    \item 实施健康中国战略  
\end{enumerate}
\subsection{应该如何形成人与自然和谐发展新格局?p239-240}
\begin{enumerate}[(1)]
    \item 把节约资源放在首位
    \item 坚持保护优先、自然恢复为主
    \item 着力推进绿色发展、循环发展、低碳发展。
    \item 形成节约资源和保护环境的空间格局、产业结构、生产方式、生活方式。
\end{enumerate}
\section{第十一章}
\subsection{“四个全面”战略布局的发展历程:p244}
党的十八大提出了到2020年全面建成小康社会的奋斗目标。
\subsection{“四个全面”战略布局是我们党在新形势下治国理政的总方略。在“四个全面”战略布局中居于引领地位的是全面建成小康社会。全面建成小康社会是实现中华民族伟大复兴的中国梦的重要基础、关键一步。p266本章小结少中间一句}
\subsection{2014年十八届四中全会通过了《关于全面推进依法治国若干重大问题的决定》,明确提出全面推进依法治国,加快建设法治中国,开启了中国特色社会主义法治道路的新征程。十九大明确提出,全面依法治国是中国特色社会主义的本质要求和重要保障。必须把党的领导贯彻落实到依法治国全过程和各方面,坚定不移走中国特色社会主义法治道路。全面依法治国,总目标是建设中国特色社会主义法治体系,建设社会主义法治国家。256}
\subsection{党的十八届三中全会通过了《中共中央关于全面深化改革若干重大问题的决定》,提出全面深化改革的总目标是完善和发展中国特色社会主义制度,推进国家治理体系和治理能力现代化。p251}
\subsection{党的十九大首次把党的政治建设纳入党的建设总体布局,并强调“以党的政治建设为统领” “把党的政治建设摆在首位”,凸显党的政治建设的极端重要性,这是党的建设理论和实践的重大创新。p261}
\subsection{改革、发展、稳定是我国社会主义现代化建设的三个重要支点,改革是经济社会发展的强大动力,发展是解决一切经济社会问题的关键,稳定是改革发展的前提。p255}
\subsection{全面建成小康社会的内涵。p244 没找到答案自己判断}
\subsection{决胜全面建成小康社会提出了哪些新要求。p248}
\begin{enumerate}[(1)]
    \item 
    \item 
\end{enumerate}
\subsection{简述我国社会主义现代化建设中的改革、发展、稳定的关系。p159老书p255新书}
\newpage
\subsection{如何正确处理全面深化改革中的重大关系p253}
\begin{enumerate}[(1)]
    \item 处理好解放思想和实事求是的关系
    \item 处理好顶层设计和摸着石头过河的关系
    \item 处理好整体推进和重点突破的关系
    \item 处理好胆子要大,步子要稳的关系
    \item 处理好改革,发展,稳定的关系。 
\end{enumerate}
\section{第十二章}
\subsection{习近平强军思想深刻回答了“新时代建设一支什么样的强大人民军队、怎样建设强大人民军队”的时代课题。p269}
\begin{enumerate}[(1)]
    \item 强国必须强军,巩固国防和强大人民军队是新时代坚持和发展中国特色社会主义、实现中华民族伟大复兴的战略支撑
    \item 党在新时代的强军目标是建设一支听党指挥、能打胜仗、作风优良的人民军队,必须同国家现代化进程相-一致,力争到2035年基本实现国防国防和军  和军队现代化,到本世纪中叶把人民军队全面建成世界一流军队。
    \item 党对军队绝对领导是人民军队建军之本、强军之魂,必须全面贯彻党领导军队的一系列根本原则和制度,确保部队绝对忠诚、绝和决心推  对纯洁、绝对可靠。
    \item 军队是要准备打仗的,必须聚焦能打仗、打胜仗,创新发展军事战略指导,构建中国特色现代作战体系,全面提高新时代备战打仗能力,有效塑造态势、管控危机、遏制战争、打赢战争。
    \item 作风优良是我军鲜明特色和政治优势,必须加强作风建设、纪律建设,坚定不移正风肃纪、反腐惩恶,大力弘扬我党我军光荣传统和优良作风,永葆人民军队性质、宗旨、本色。
    \item 推进强军事业必须坚持政治建军、改革强军、科技兴军、依法治军,更加注重聚焦实战、更加注重创新驱动、更加注重体系建设、更加注重集约高效、更加注重军民融合,全面提高革命化现代化正规化革命水平。
    \item 是改革是强军的必由之路,必须推进军队组织形态现代化,邓小  构建中国特色现代军事力量体系,完善中国特色社会主义军事制度。
    \item 创新是引领发展的第一动力,必须坚持向科技创新要战斗力,统筹推进军事理论、技术、组织、管理、文化等各方面创新,建设创新型人民军队
    \item 现代化军队必须构建中国特色军事法治体系,推进治军方式根本性转变,提高因防和军队建设法治化水平。
    \item 军民融合发展是兴国之举、强军之策,必须坚持发展和安全兼顾、富国和强军统一,形成全要素、多领域、高效益军民融合深度发展格局,构建体化的国家战略体系和能力。
\end{enumerate}
\subsection{党对军队绝对领导是中国特色社会主义的本质特征,是党和国家的重要政治优势。p270}
\subsection{党在新时代的强军目标及路线图。p272}
\par 党在新时代的强军目标是建设一支听党指挥、能打胜仗、作风优良的人民军队,必须同国家现代化进程相一致。到2020年,国防和军队建设基本实现机械化,信息化建设取得重大进展,战略能力有大的力争提升;到2035年基本实现国防和军队现代化,到本世纪中叶把人民军队全面建成世界一流军队。
\subsection{走军民融合式发展之路,是实现富国和强军相统一的重要途径。p278}
\subsection{牢固树立战斗力唯一的标准。p273}
\subsection{政治建军是人民军队立军之本。改革是我军发展壮大、致胜未来的关键一招。科技是现代战争的核心战斗力。依法治军、从严治军是我们党建军治军的基本方略。p273-275第一句}
\section{第十三章}
\subsection{和平与发展仍然是时代主题,和平、发展、合作、共赢成为不可阻挡的时代潮流。}
\subsection{世界多极化、经济全球化、文化多样化、社会信息化深入发展,全球治理体系和国际秩序变革加速推进,各国相互联系和依存日益加深,国际力量对比更趋平衡,和平发展大势不可逆转。}
\subsection{世界正处于大发展大变革大调整的表现:世界多极化在曲折中发展; 经济全球化深入发展;文化多样化持续推进;社会信息化快速发展;科学技术孕育新突破。}
\subsection{中国形成了独立自主的和平外交政策,成功地走上了一条与本国国情和时代特征相适应的和平发展道路。}
中国坚定不移地奉行独立自主的和平外交政策,是由我国的社会主义性质和在国际上的地位所决定的,是从历史、现实、未来的客观判断中得出的结论,是思想自信和实践自觉的有机统一。中国走和平发展道路的自信和自觉,来源于中华文明的深厚渊源,来源于对实现中国发展目标条件的认知,来源于对世界发展大势的把握。
\subsection{维护世界和平、促进共同发展,是中国外交政策的宗旨。推动建设相互尊重、公平正义、合作共赢的新型国际关系。}
\subsection{推动建立新型国际关系的要求:p287-288}
推动建立新型国际关系,要坚决维护国家核心利益;推动建立新型国际关系,要坚决维护国家核心利益。要按照大国是关键、周边是首要、发展中国家是基础、多边是舞台的外交工作布局,建立新型国际关系。推动建立新型国际关系,要加强涉外法律工作,完善涉外法律法规体系。推动建立新型国际关系,要积极参与全球治理体系改革和建设。推动建立新型国际关系,还要把合作共赢理念体现到政治、经济、安全、文化等对外合作的方方面面,推动构建人类命运共同体。
\subsection{构建人类命运共同体思想,是一个科学完整、内涵丰富、意义深远的思想体系,其核心就是“建设持久和平、普遍安全、共同繁荣、开放包容、清洁美丽的世界”。p290-292}
第一,政治上,要相互尊重、平等协商,坚决摒弃冷战思维和强权政治,走对话而不对抗、结伴而不结盟的国与国交往新路。第二,安全上,要坚持以对话解决争端、以协商化解分歧,统筹应对传统和非传统安全威胁,反对一切形式的恐怖主义。第三,经济上,要同舟共济,促进贸易和投资自由化便利化,推动经济全球化朝着更加开放、包容、普惠、平衡、共赢的方向发展。第四,文化上,要尊重世界文明多样性,以文明交流超越文明隔阂、文明互鉴超越文明冲突、文明共存超越文明优越。第五,生态上,要坚持环境友好,合作应对气候变化,保护好人类赖以生存的地球家园。
\subsection{如何共商共建人类命运共同体。p294-295}
第一,坚持和平发展道路,推动建设新型国际关系。第二,不断完善外交布局,打造全球伙伴关系网络。第三,深度参与全球治理,积极引导国际秩序变革方向。第四,推动国际社会从伙伴关系、安全格局、经济发展、文明交流、生态建设等方面为建立人类命运共同体作出努力。
\subsection{习近平提出共建“丝绸之路经济带”和“21世纪海上丝绸之路”的重大倡议。p294}
的十九大对“一带一路”建设的规划,坚持引进来和走出去并重,遵循共商共建共享原则,加强创新能力开放合作,形成陆海内外联动、东西双向互济的开放格局。一是要坚持引进来和走出去并重,深化双向投资合作。。二是坚持共商共建共享原则。三是加强创新能力开放合作,主要是加强技术创新合作、理论创新交流互鉴、创新人才资源交流合作。四是把“一带一路”与构建人类命运共同体更加紧密结合起来,与落实2030年可持续发展议程紧密结合起来,打造国际合作新平台,增添共同发展新动力,把“一带一路”建成和平之路、繁荣之路、开放之路、创新之路、文明之路。
\section{第十四章}
\subsection{《中国共产党党章》规定:中国共产党是工人阶级的先锋队,同时是中国人民和中华民族的先锋队,是中国特色社会主义事业的领导核心,代表中国先进生产力的发展要求,代表中国先进文化的前进方向,代表中国最广大人民的根本利益。p297}
\subsection{党的领导是中国特色社会主义最本质的特征,是中国特色社会主义制度的最大优势。p298-299}
十三届全国人大一次会议审议通过的宪法修正案,把“中国共产党的领导是中国特色社会主义最本质的特征”载入宪法总纲,以国家根本大法的形式强调党的领导在中国特色社会主义中的核心地位。
\subsection{新时代中国共产党的历史使命,就是统揽伟大斗争、伟大工程、伟大事业、伟大梦想,在全面建成小康社会的基础上全面建成社会主义现代化强国,实现中华民族伟大复兴的中国梦。p301}
\subsection{“四个伟大”的关系。其中,起决定性作用的是什么。p303}
党的建设
\subsection{为什么党是最高政治领导力量。p304-305}
是马克思主义政党的基本要求,是对党领导革命、建设和改革历史经验的深刻总结,是推进伟大事业的根本保证。
\subsection{如何确保党始终纵览全局、协调各方。p305-306}
\begin{enumerate}[(1)]
    \item 增强政治意识,大局意识,核心意识,看齐意识自觉维护党中央权威和集中统一领导,自觉在思想上政治上行动上同党中央保持高度一致
    \item 必须坚持和完善党的领导的体制机制
    \item 必须坚持党的民主集中制原则
    
\end{enumerate}
\subsection{“四个意识”的内容p305}
政治意识、大局意识、核心意识、看齐意识
\subsection{如何全面增强党的执政本领。p307}
\begin{enumerate}[(1)]
    \item 增强学习本领
    \item 增强政治领导本领
    \item 增强改革创新本领
    \item 增强科学发展本领
    \item 增强依法执政本领
    \item 增强群众工作本领
    \item 增强狠抓落实本领
    \item 增强驾驭风险本领
\end{enumerate}
\clearpage 
\section*{声明}
\par  本文主要作为复习参考,由于时间有限,并没有把内容写全面(真的心累),严禁本文作为小抄缩印
\par  本文主要采用texlive2017+VS code+xelatex 方式编写,代码量450行
\par  编程主要代码已开源,时间有限,有错误,私聊我。
\par  毛概已经弄出来,非常感谢小刚的帮助,同时希望大家有一个好的成绩。
\par  原创不易,希望大家多多支持。希望大家喜欢的话可以进行捐助,金额不限,我就是皮一下。
\par  代码地址  https://github.com/CherryRum/Course-/tree/master/maogai  
\par  使用代码时注释掉图片
\centerline{\includegraphics[width=6cm,height=9cm]{9.jpg}}
\end{document}
