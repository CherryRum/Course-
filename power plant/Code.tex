\documentclass[UTF8]{ctexart}
\usepackage{fancyhdr}
\usepackage{titlesec}
\usepackage{titletoc}
\usepackage{appendix}
\usepackage{bm, amsmath,amsfonts}
\usepackage{multirow}
\usepackage{enumerate}
\usepackage[a4paper,left=3.4cm,right=3cm,top=1.65cm,bottom=2.54cm]{geometry}
\renewcommand{\contentsname}{\zihao{3} 目\quad 录}

%页眉页脚设置
\pagestyle{fancy}
\fancyhf{}
\cfoot{\thepage}
\lhead{\kaishu~发电厂期末复习~}
\rhead{\kaishu~电气153}

%目录页设置
\titlecontents{section}[0em]{\zihao{4}\bf }{\thecontentslabel\ }{}
{\hspace{.5em}\titlerule*[4pt]{$\cdot$}\contentspage}
\titlecontents{subsection}[2em]{\vspace{0.1\baselineskip}\zihao{-4}}{\thecontentslabel\ }{}
{\hspace{.5em}\titlerule*[4pt]{$\cdot$}\contentspage}
\titlecontents{subsubsection}[4em]{\vspace{0.1\baselineskip}\zihao{-4}}{\thecontentslabel\ }{}
{\hspace{.5em}\titlerule*[4pt]{$\cdot$}\contentspage}

%------------------  -----------------------------------------------------
%正文部分
\begin{document}
\tableofcontents\thispagestyle{empty}
\newpage
\setcounter{page}{1}
\section{第一章}  
\subsection{电能特点}
\begin{enumerate}[(1)]
\item  电能可以大规模生产和远距离输送。
\item  电能方值转换和易于控制.
\item  损耗小.  
\item  效率高。  
\item  电能在使用时没有污染,噪声小。
\end{enumerate}
\subsection{发电厂}
将各种一次能源转变成电能的工厂,称为发电厂。
\subsection{火力转换}
\par 燃料的化学能--热能--机械能--电能
\subsection{水力转换}
\par 水的势能和动能转换成电能
\subsection{抽水蓄能电厂在电力系统中的作用}
调峰、填谷、事故备用、调频、调相、黑启动、蓄能
\subsection{变电厂的分类}
\begin{enumerate}[1]
\item  按变电站在电力系统中的地位和作用分类
\begin{enumerate}[(1)]
 \item 枢纽变电站
 \item 中间变电站
 \item 区域(地区)变电站
 \item 企业变电站
 \item 末端(用户)变电站   
\end{enumerate}
\item 按变电站建筑形势和电气设备布置方式分类
\begin{enumerate}[(1)]
  \item 户内变电站
  \item 半户内变电站
  \item 户外变电站  
\end{enumerate}  
\end{enumerate}
\subsection{二次设备}
对一次设备和系统的运行状态进行测量、控制、监视、和起保护作用的设备,称为二次设备
\section{第二章}
\subsection{导体}
热量的耗散有对流、辐射、和导热三种方式
\subsection{减少钢构件损耗和发热}
\begin{enumerate}[(1)]
  \item 加大钢结构和导体之间的距离,使磁场强度减弱,因此可降低涡流和磁滞损耗
  \item 断开钢构件回路,并加上绝缘垫,消除环流
  \item 采用电磁屏蔽
  \item 采用分相封闭母线 
\end{enumerate}
\subsection{短路电流最大值}
t = 0.01s 短路电流的幅值最大,冲击电流 $i_{sh}^{(3)} = 1.82I_{m}$。
\subsection{电弧的形成}
电弧的形成与维持通常经过电子发射、碰撞游离和热游离
\subsection{电弧间隙的去游离}
\begin{enumerate}[1]
   \item 复合是指正离子和负离子互相吸引,结合在一起,电荷互动中和的过程
   \item 扩散是指带电质点从电弧内部逸出而进入周围介质中的现象。
   \par \quad 浓度扩散:
   \par \qquad 带电质点将会由浓度高的弧道周围扩散,使弧道中带点质电减少。 
   \par \quad 温度扩散:
   \par \qquad 弧道中的高温带电质点向温度低的周围介质扩散。 
\end{enumerate}
\section{第三章}
\subsection{电弧的特性几灭弧都基本原理}
\par 交流电弧具有过零值自然熄灭及动态的伏安特性。
\par 决定交流电弧熄弧的基本因素是弧隙的介质强度恢复过程和加在弧隙上的电压恢复过程。
\subsection{油断路器的分类}
\par 油断路器采用变压器油作为灭弧介质。其按绝缘结构可分为多油和少油断路器两大类。
\par \quad 多油式断路器中的油具有灭弧和绝缘两大功能,因此外 不带电压。
\par \quad 少油断路器中的油仅作为灭弧介质使用,因此其外壳带有高压,漆为红色。
\section{第四章}
\subsection{主接线基本形式,图4-3、4-4}
\par 有汇流母线分为单母线接线和双母线接线,无汇流母线分为桥式接线、角形接线和单元接线。
\subsection{限制短路电流的方法}
\begin{enumerate}[(1)]
\item 装设限流电抗器
\item 采用低压分裂绕组变压器
\item 采用不同的主接线形式和运行方式    
\end{enumerate}
\section{第五章}
\subsection{厂用电源}
\begin{enumerate}[1]
    \item 工作电源
    \item 备用电源和启动电源
    \par \quad 备用电通有明备用和时备用两种方式。
    \par \qquad 明备用方式,正如前面所述设置专用的备用变压件(或线路),也经常处齐用状态(停运), 当工作电源因故断开时,由能讲电源白动投人装置进行切换接通,代替工作电源,承担全部厂用负荷。
    \par \qquad 暗备用方式,不设专用的备用变压器(或线路),而将征台干作变压都在旅增大,相互备用,当其中任一台厂 用工作变压器退出达行时,该台工作业乐器所承担负荷由另一台厂用工作安压器供电,这种备用方式,正常工作时每台变压器只在半载下运行,投资较大、运行费用高,
    \item 事故保安电源  
\end{enumerate}
\subsection{厂用电接线形式}
\par 发电厂厂用电系统接线通常都采用单母线分断接线形式,并多以成套配电装置接受和分配电能。
\par 厂用电各级电压母线均采用按锅炉分段接线方式。
\section{第六章}
\subsection{发电机断路器的特殊要求}
\begin{enumerate}[(1)]
    \item 额定值方面的要求
    \item 开断性能方面的要求
    \item 固有恢复电压方面的要求
\end{enumerate}
\subsection{隔离开关的特点}
\begin{enumerate}[(1)]
    \item 隔离电压
    \item 倒闸操作
    \item 分、合小电流
\end{enumerate}
\subsection{互感器}
\par 电流互感器在运行时,二次绕组严禁开路。
\subsection{导体截面积选择}
\begin{enumerate}[1.]
    \item 按导体长期发热允许电流选择
    \item 按经济电流密度选择    
\end{enumerate}
\section{第七章}
\subsection{配电装置的类型}
\begin{enumerate}[(1)]
    \item 屋内配电装置的特点
    \begin{enumerate}[(1)]
        \item 由于允许安全净距小和可以分层布置而使占地面积较小
        \item 维修、巡视和操作在室内进行,可减轻维护工作量,不受气候影响
        \item 外界污秽空气对电器影响较小,可以减小维护工作量
        \item 房屋建筑投资较大,建设周期长,但可采用价格较低的户内型设备
    \end{enumerate}
    \item 成套配电装置的特点
    \begin{enumerate}[(1)]
        \item 电器布置在封闭或半封闭的金属中,相间和对地距离可以缩小,结构紧凑,占地面积小
        \item 所有电器元件已在工厂组装成一体,如$SF-{6}$全封闭组合电器、开关柜等,大大减少现场安装工作量,有利于缩短建设周期,也便于扩建和拆迁
        \item 运行可靠性高,维护方便
        \item 耗用钢材多、造价较高
    \end{enumerate}
\end{enumerate}
\clearpage 
\section*{声明}
\par  本文主要作为复习参考,由于时间有限,并没有把内容加入例题,严谨本文作为小炒缩印
\par  本文主要采用texlive2017+VS code 编程主要代码就开源了吧,时间有限,有错误,私聊我。
\par  毛概的正在弄,但是可能出不来,时间有点少,我尽量弄吧,希望大家有一个好的成绩。
\par  原创不易,希望大家多多支持。
\end{document}
